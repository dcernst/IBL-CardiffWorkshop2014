\documentclass[10pt,handout]{beamer}

\usepackage{spot}

\usepackage{tikz}
\usetikzlibrary{shapes}
\usetikzlibrary{decorations.pathmorphing} 
\usepackage{color}

\usetheme{Pittsburgh}
\usecolortheme{seahorse}
\useoutertheme{split}

\setbeamertemplate{footline}[split, frame number]
\setbeamertemplate{enumerate items}[default]
\setbeamertemplate{itemize items}[circle]

\newtheorem{comments}{Comments}
\newtheorem{question}{Question}
\newtheorem{goal}{Goal}
\newtheorem{remark}{Remark}
\newtheorem{proposition}{Proposition}
\newtheorem{conjecture}{Conjecture}

\newcommand*\oldmacro{}
\let\oldmacro\insertshorttitle
\renewcommand*\insertshorttitle{
\oldmacro\hfill
\insertframenumber\,/\,\inserttotalframenumber}

%% ----------------------------------------------------------------------  

\begin{document}

\setspotlightstyle{rectangle, rounded corners,fill=structure.fg!25!white,path fading=none}

\title[Inquiry-Based Education in Mathematics]
{\large \textbf{Inquiry-Based Education in Mathematics: Models, Methods, \& Effectiveness for Higher Education}}
\author[D.C.~Ernst and TJ Hitchman]{Dana C.~Ernst, Northern Arizona University\\
Theron Hitchman, University of Northern Iowa}
\institute{\url{http://danaernst.com}\\
\url{http://www.uni.edu/theron/}}

\vspace{1em}

\date{\textbf{Workshop on Innovations in Higher\\ Education Mathematics Teaching}\\
Cardiff University, 7--9 July 2014}

\frame{\titlepage}

%% ----------------------------------------------------------------------

\begin{frame}

\vfill
\begin{center}
\spot[inner sep=3.5ex]{{\Huge \textbf{What is IBL?}}}
\end{center}
\vfill

\end{frame}

%% ----------------------------------------------------------------------

\begin{frame}

\begin{block}{The Big Picture}

\vspace{1em}

\spot[inner sep=1ex]{{
\begin{quote}
\textbf{If we really want students to be independent, inquisitive, \& persistent, then we need to provide them with the means to acquire these skills.}
\end{quote}
}}
\end{block}

\end{frame}

%% ----------------------------------------------------------------------

\begin{frame}

\begin{block}{What is inquiry-based learning (IBL)?}
\begin{itemize}
\item Hard to define! Manifests itself differently in different contexts.
\item According to the \alert{Academy of Inquiry-Based Learning}:
\begin{itemize}\normalsize
\item IBL is a teaching method that engages students in sense-making activities.
\item Students are given tasks requiring them to solve problems, conjecture, experiment, explore, create, & communicate.
\item Rather than showing facts and/or algorithms, the instructor guides students via well-crafted problems.
\end{itemize}
\item Often involves very little lecturing, and typically involves student presentations.
\item Example: Modified Moore Method, after R.L. Moore.
\item Students should as much as possible be responsible for:
\begin{itemize}\normalsize
\item guiding the acquisition of knowledge and,
\item validating the ideas presented. (Students should not be looking to the instructor as the sole authority.)
\end{itemize}
\end{itemize}
\end{block}

\end{frame}

%% ----------------------------------------------------------------------

\begin{frame}

\begin{block}{Guiding Principle of IBL}
Continually ask yourself the following question:

\vspace{1em}

\spot[inner sep=1ex]{ 
\begin{quote}
\textbf{Where do I draw the line between content I must impart to my students versus content they can produce independently?}
\end{quote}}

\end{block}

\begin{block}{Our Main Objective}
How do we get here?

\begin{center}
\tikzstyle{block} = [rectangle, fill=structure.fg!25!white, text width=6em, text centered, rounded corners, minimum height=2.5em,inner sep=1ex]
\tikzstyle{line} = [draw, very thick, -stealth]
\begin{tikzpicture}
\node [block] (answer) at (0,0) {\textbf{Students answering questions}};
\node [block] (ask) at (4,0) {\textbf{Students asking questions}};
\path [line] (answer) -- (ask);
\end{tikzpicture}
\end{center}
\end{block}

\end{frame}

%% ----------------------------------------------------------------------

\begin{frame}

\begin{block}{Two Typical Approaches/Modes to IBL}
\begin{enumerate}
\item Student presentations.
\item Small group work.
\end{enumerate}
Most IBL instructors implement some combination.
\end{block}

\begin{block}{IBL vs Presentations/Group Work}
\begin{itemize}
\item Student presentations \& group work act as vehicles for IBL. 
\item Yet student presentations \& group do not imply IBL. 
\item What matters is what is happening during these activities.
\end{itemize}
\end{block}

\end{frame}

%% ----------------------------------------------------------------------

\begin{frame}

\begin{block}{IBL vs Inverted/Flipped Pedagogy}
\begin{itemize}
\item IBL/Moore Method is an instructional practice.
\item The flipped classroom is:
    \begin{itemize}\normalsize
    \item A platform, not an instructional practice. 
    \item Centered around the idea of removing some/all of the information transfer tasks outside of class & replacing the time that’s freed up with whatever instructor feels is appropriate.
    \end{itemize}
\end{itemize}
\end{block}

\end{frame}

%% ----------------------------------------------------------------------

\end{document}