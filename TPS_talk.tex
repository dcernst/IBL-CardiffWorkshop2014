\documentclass[10pt,handout]{beamer}

\usepackage{spot}
\usepackage{color}

\usetheme{Pittsburgh}
\usecolortheme{seahorse}
\useoutertheme{split}

\setbeamertemplate{footline}[split, frame number]
\setbeamertemplate{enumerate items}[default]
\setbeamertemplate{itemize items}[circle]

\newtheorem{comments}{Comments}
\newtheorem{question}{Question}
\newtheorem{goal}{Goal}
\newtheorem{remark}{Remark}
\newtheorem{proposition}{Proposition}
\newtheorem{conjecture}{Conjecture}

\newcommand*\oldmacro{}
\let\oldmacro\insertshorttitle
\renewcommand*\insertshorttitle{
\oldmacro\hfill
\insertframenumber\,/\,\inserttotalframenumber}

%% ----------------------------------------------------------------------  

\begin{document}

\setspotlightstyle{rectangle, rounded corners,fill=structure.fg!25!white,path fading=none}

\title[Inquiry-Based Education in Mathematics]
{\large \textbf{Inquiry-Based Education in Mathematics: Models, Methods, \& Effectiveness for Higher Education}}
\author[D.C.~Ernst and TJ Hitchman]{Dana C.~Ernst, Northern Arizona University\\
Theron J.~Hitchman, University of Northern Iowa}
\institute{\url{http://danaernst.com}\\
\url{http://www.uni.edu/theron/}}

\vspace{1em}

\date{\textbf{Workshop on Innovations in Higher\\ Education Mathematics Teaching}\\
Cardiff University, 7--9 July 2014}

\frame{\titlepage}

\begin{frame}

\vfill
\begin{center}
\spot[inner sep=3.5ex]{{\Huge \textbf{Introductory Questions}}}
\end{center}
\vfill

\end{frame}

%% ----------------------------------------------------------------------

\begin{frame}{\textbf{Question One}}

\begin{center}{\Huge Why are we here?}\end{center}

\end{frame}

\begin{frame}{\textbf{Question One: Sharper Version}}

\begin{center}{\Huge From a learner's perspective, what is the purpose of continuing to study past secondary school?}\end{center}

\end{frame}

\begin{frame}{\textbf{Question Two}}

\begin{center}{\Huge What are the goals of a university education?}\end{center}

\end{frame}

\begin{frame}{\textbf{Question Three}}

\begin{center}{\Huge Information is very free and open these days. Given that one can read and study on one's own, what is the point of going to university?}\end{center}

\end{frame}


\begin{frame}{\textbf{Question Four}}

\begin{center}{\Huge What do you reasonably expect that your students will remember from your courses 20 years later?}\end{center}

\end{frame}

\begin{frame}{\textbf{Question Five}}

\begin{center}{\Huge How does a person learn something new?}\end{center}

\end{frame}

\begin{frame}{\textbf{Question Six}}

\begin{center}{\Huge How should a course of study be structured to facilitate learning?}\end{center}

\end{frame}

\begin{frame}{\textbf{Question Seven}}

{\Large
What are the potential pitfalls? What challenges do we face in building a community where we all engage in collaborative inquiry?}\\[.5in]

Specifically:
\begin{center}{\huge What barriers would a newcomer encounter that might make them leave such a group?}\end{center}

\end{frame}

\begin{frame}{\textbf{Question Eight}}

\begin{center}{\Huge How do we create a safe space, so that we can all engage in the process with minimum psychological damage?}\end{center}

\end{frame}

%% ----------------------------------------------------------------------

\end{document}