\documentclass[12pt]{article}

\usepackage{url}
\usepackage[metapost,truebbox]{mfpic}
\usepackage{tikz}
\usepackage{fancyhdr}
\usepackage[margin=1in]{geometry}
\usepackage[hang,flushmargin,symbol*]{footmisc}
\usepackage{amsmath}
\usepackage{amsthm}
\usepackage{todonotes}
\usepackage{amssymb}
\usepackage{mathtools}
\usepackage{enumitem}
\usepackage{graphicx}
\usepackage{caption}
\usepackage{subcaption}
\usepackage{float}
\usepackage{color}
\usepackage{tipa} %to get \textpipe to work
\definecolor{darkblue}{rgb}{0, 0, .6}
\definecolor{grey}{rgb}{.7, .7, .7}
\usepackage[breaklinks]{hyperref}
\hypersetup{
	colorlinks=true,
	linkcolor=darkblue,
	anchorcolor=darkblue,
	citecolor=darkblue,
	pagecolor=darkblue,
	urlcolor=darkblue,
	pdftitle={},
	pdfauthor={}
}


%%%%%%Header/Footer%%%%%%%

\pagestyle{fancy}

\lhead{\scriptsize Introduction to Inquiry-Based Learning} 
\chead{} 
\rhead{\scriptsize Engaging in Dialogue with Students} 
\lfoot{\scriptsize } 
\cfoot{} 
\rfoot{\scriptsize Written by \href{http://danaernst.com}{D.C. Ernst}} 
\renewcommand{\headrulewidth}{0.4pt} 
\renewcommand{\footrulewidth}{0.4pt} 

%%%%%%%%%%%%%%%%%%%

\begin{document}

\begin{center}

{\Large\bf Engaging in Dialogue with Students\footnote{This exercise is borrowed from the \href{http://www.artofmathematics.org/}{Discovering the Art of Mathematics} project.}} 

\end{center}

\setlength{\parindent}{0pt}
\setlength{\fboxsep}{10pt}

Suppose you are the instructor of a second semester calculus class and you have chosen to have students work in small groups to explore the notions of convergence and divergence of series.  At this point the students have not been provided with these definitions, but you have given them some guided exercises to play with.

While one group is exploring the series $1+\frac{1}{2}+\frac{1}{4}+\frac{1}{8}+\cdots$, you overhear the following conversation.

\begin{itemize}
\item[] \textbf{Max:} Well, all the numbers are added so it will be more and more. I think it will never stop!
\item[] \textbf{Tina:} But aren't they getting smaller and smaller? Like 1/2 is less than 1 and 1/4 is less than 1/2...
\item[] \textbf{Max:} Yes, but I am adding them up, see? (Pointing to the plus sign between the numbers.)
\item[] \textbf{Tina:} Hmmm (clearly thinking hard). Okay, I agree, I add them up, but they are still getting smaller and smaller, so it should stop sometime, right?
\item[] \textbf{Max:} It can't stop, since there are always more numbers being added, the numbers never stop. So I still think it will never stop.
\item[] \textbf{Kim:} So wait, you can't both be right, so who is right now? I am so confused!
\end{itemize}

\textbf{Task:}  Now you have to decide what to say to them.  Consider the following responses.  With a buddy or two, discuss the advantages and disadvantages for each.

\begin{enumerate}
\item Max is wrong.  I'll let you figure out why.
\item I think Tina is onto something.
\item So, you are saying that the numbers we are adding never stop.  Do you think that the sums will stop changing?  If not, is there a limit that the sums will approach?
\item Can you draw a picture?
\item So, Max, what's the ``it" you are talking about?
\end{enumerate}

How would you ultimately respond?

\end{document}